% Options for packages loaded elsewhere
\PassOptionsToPackage{unicode}{hyperref}
\PassOptionsToPackage{hyphens}{url}
%
\documentclass[
]{article}
\usepackage{amsmath,amssymb}
\usepackage{lmodern}
\usepackage{iftex}
\ifPDFTeX
  \usepackage[T1]{fontenc}
  \usepackage[utf8]{inputenc}
  \usepackage{textcomp} % provide euro and other symbols
\else % if luatex or xetex
  \usepackage{unicode-math}
  \defaultfontfeatures{Scale=MatchLowercase}
  \defaultfontfeatures[\rmfamily]{Ligatures=TeX,Scale=1}
\fi
% Use upquote if available, for straight quotes in verbatim environments
\IfFileExists{upquote.sty}{\usepackage{upquote}}{}
\IfFileExists{microtype.sty}{% use microtype if available
  \usepackage[]{microtype}
  \UseMicrotypeSet[protrusion]{basicmath} % disable protrusion for tt fonts
}{}
\makeatletter
\@ifundefined{KOMAClassName}{% if non-KOMA class
  \IfFileExists{parskip.sty}{%
    \usepackage{parskip}
  }{% else
    \setlength{\parindent}{0pt}
    \setlength{\parskip}{6pt plus 2pt minus 1pt}}
}{% if KOMA class
  \KOMAoptions{parskip=half}}
\makeatother
\usepackage{xcolor}
\usepackage[margin=1in]{geometry}
\usepackage{color}
\usepackage{fancyvrb}
\newcommand{\VerbBar}{|}
\newcommand{\VERB}{\Verb[commandchars=\\\{\}]}
\DefineVerbatimEnvironment{Highlighting}{Verbatim}{commandchars=\\\{\}}
% Add ',fontsize=\small' for more characters per line
\usepackage{framed}
\definecolor{shadecolor}{RGB}{248,248,248}
\newenvironment{Shaded}{\begin{snugshade}}{\end{snugshade}}
\newcommand{\AlertTok}[1]{\textcolor[rgb]{0.94,0.16,0.16}{#1}}
\newcommand{\AnnotationTok}[1]{\textcolor[rgb]{0.56,0.35,0.01}{\textbf{\textit{#1}}}}
\newcommand{\AttributeTok}[1]{\textcolor[rgb]{0.77,0.63,0.00}{#1}}
\newcommand{\BaseNTok}[1]{\textcolor[rgb]{0.00,0.00,0.81}{#1}}
\newcommand{\BuiltInTok}[1]{#1}
\newcommand{\CharTok}[1]{\textcolor[rgb]{0.31,0.60,0.02}{#1}}
\newcommand{\CommentTok}[1]{\textcolor[rgb]{0.56,0.35,0.01}{\textit{#1}}}
\newcommand{\CommentVarTok}[1]{\textcolor[rgb]{0.56,0.35,0.01}{\textbf{\textit{#1}}}}
\newcommand{\ConstantTok}[1]{\textcolor[rgb]{0.00,0.00,0.00}{#1}}
\newcommand{\ControlFlowTok}[1]{\textcolor[rgb]{0.13,0.29,0.53}{\textbf{#1}}}
\newcommand{\DataTypeTok}[1]{\textcolor[rgb]{0.13,0.29,0.53}{#1}}
\newcommand{\DecValTok}[1]{\textcolor[rgb]{0.00,0.00,0.81}{#1}}
\newcommand{\DocumentationTok}[1]{\textcolor[rgb]{0.56,0.35,0.01}{\textbf{\textit{#1}}}}
\newcommand{\ErrorTok}[1]{\textcolor[rgb]{0.64,0.00,0.00}{\textbf{#1}}}
\newcommand{\ExtensionTok}[1]{#1}
\newcommand{\FloatTok}[1]{\textcolor[rgb]{0.00,0.00,0.81}{#1}}
\newcommand{\FunctionTok}[1]{\textcolor[rgb]{0.00,0.00,0.00}{#1}}
\newcommand{\ImportTok}[1]{#1}
\newcommand{\InformationTok}[1]{\textcolor[rgb]{0.56,0.35,0.01}{\textbf{\textit{#1}}}}
\newcommand{\KeywordTok}[1]{\textcolor[rgb]{0.13,0.29,0.53}{\textbf{#1}}}
\newcommand{\NormalTok}[1]{#1}
\newcommand{\OperatorTok}[1]{\textcolor[rgb]{0.81,0.36,0.00}{\textbf{#1}}}
\newcommand{\OtherTok}[1]{\textcolor[rgb]{0.56,0.35,0.01}{#1}}
\newcommand{\PreprocessorTok}[1]{\textcolor[rgb]{0.56,0.35,0.01}{\textit{#1}}}
\newcommand{\RegionMarkerTok}[1]{#1}
\newcommand{\SpecialCharTok}[1]{\textcolor[rgb]{0.00,0.00,0.00}{#1}}
\newcommand{\SpecialStringTok}[1]{\textcolor[rgb]{0.31,0.60,0.02}{#1}}
\newcommand{\StringTok}[1]{\textcolor[rgb]{0.31,0.60,0.02}{#1}}
\newcommand{\VariableTok}[1]{\textcolor[rgb]{0.00,0.00,0.00}{#1}}
\newcommand{\VerbatimStringTok}[1]{\textcolor[rgb]{0.31,0.60,0.02}{#1}}
\newcommand{\WarningTok}[1]{\textcolor[rgb]{0.56,0.35,0.01}{\textbf{\textit{#1}}}}
\usepackage{graphicx}
\makeatletter
\def\maxwidth{\ifdim\Gin@nat@width>\linewidth\linewidth\else\Gin@nat@width\fi}
\def\maxheight{\ifdim\Gin@nat@height>\textheight\textheight\else\Gin@nat@height\fi}
\makeatother
% Scale images if necessary, so that they will not overflow the page
% margins by default, and it is still possible to overwrite the defaults
% using explicit options in \includegraphics[width, height, ...]{}
\setkeys{Gin}{width=\maxwidth,height=\maxheight,keepaspectratio}
% Set default figure placement to htbp
\makeatletter
\def\fps@figure{htbp}
\makeatother
\setlength{\emergencystretch}{3em} % prevent overfull lines
\providecommand{\tightlist}{%
  \setlength{\itemsep}{0pt}\setlength{\parskip}{0pt}}
\setcounter{secnumdepth}{-\maxdimen} % remove section numbering
\ifLuaTeX
  \usepackage{selnolig}  % disable illegal ligatures
\fi
\IfFileExists{bookmark.sty}{\usepackage{bookmark}}{\usepackage{hyperref}}
\IfFileExists{xurl.sty}{\usepackage{xurl}}{} % add URL line breaks if available
\urlstyle{same} % disable monospaced font for URLs
\hypersetup{
  pdftitle={Reltrad in the U.S., 1972-2018},
  pdfauthor={David Eagle, PhD},
  hidelinks,
  pdfcreator={LaTeX via pandoc}}

\title{Reltrad in the U.S., 1972-2018}
\author{David Eagle, PhD}
\date{2019-01-15}

\begin{document}
\maketitle

Religious affliation has changed significantly in the United States
since 1972. In this example, I plot the change in religious affiliation
over time using data from the 1972--2018
\href{https://gss.norc.org/get-the-data}{General Social Survey}.

I have code on \href{https://github.com/thebigbird/reltrad}{Github} that
recodes the GSS into the standard RELTRAD categories that are frequently
used by sociologist of religion. This code shows how to use ggplot to
create a plot of the trends over time
\[\@steenslandMeasureAmericanReligion2000\].

\texttt{\{r\ setup,\ include=FALSE\}\ knitr::opts\_chunk\$set(echo\ =\ TRUE,\ message\ =\ F)}

\begin{Shaded}
\begin{Highlighting}[]
\NormalTok{\#Create a Figure for GSS attendance over time}
\NormalTok{\#Plot the proportions over time}
\NormalTok{source("https://raw.githubusercontent.com/thebigbird/R\_Stata\_Reltrad/master/ReltradGSS.R")}
\NormalTok{\#See my github for the code to make these data}

\NormalTok{\#Cut by generation}
\NormalTok{gss = gss \%\textgreater{}\% }
\NormalTok{  mutate(year = as.numeric(as.character(year))) \%\textgreater{}\%}
\NormalTok{  mutate(birthyr = year {-} age) \%\textgreater{}\%}
\NormalTok{  mutate(age\_cat = cut(birthyr, c(0,1928,1945,1965,1981,1997,2020)))}

\NormalTok{levels(gss$age\_cat) = c("Great","Silent","Boomer","GenX","Millenial","GenZ")}

\NormalTok{library(srvyr)}
\NormalTok{\#Calculate weighted proportions}
\NormalTok{\#Create survey object using svyr}
\NormalTok{gss\_svy \textless{}{-} gss \%\textgreater{}\% as\_survey\_design(1, weight = wtssall,}
\NormalTok{                                    variables = c(year, reltrad, age\_cat))}

\NormalTok{\#Now calculate proportions}
\NormalTok{out = gss\_svy \%\textgreater{}\% }
\NormalTok{  group\_by(year, reltrad) \%\textgreater{}\%}
\NormalTok{  summarize(prop = survey\_mean(na.rm=T, proportion = T)) \%\textgreater{}\%}
\NormalTok{  drop\_na()}

\NormalTok{\#A nice set o\textquotesingle{} colors}
\NormalTok{scFill = scale\_color\_manual(values = }
\NormalTok{                              c("\#1B9E77", }
\NormalTok{                                "\#DDC849",}
\NormalTok{                                "\#7570b3", }
\NormalTok{                                "\#a6761d",}
\NormalTok{                                "\#e7298a",}
\NormalTok{                                "\#1f78b4",}
\NormalTok{                                "\#a9a9a9",}
\NormalTok{                                "\#cc99ff"))}

\NormalTok{ggplot(data = out, }
\NormalTok{                 aes(x=year, }
\NormalTok{                     y=prop,}
\NormalTok{                     color = reltrad,}
\NormalTok{                     group = reltrad)) +}
\NormalTok{  geom\_ribbon(aes(ymin = prop {-} 1.96*prop\_se,}
\NormalTok{                  ymax = prop + 1.96*prop\_se),}
\NormalTok{                  color = "white",}
\NormalTok{                  alpha = .2, fill = "grey") +}
\NormalTok{  geom\_point() +}
\NormalTok{  geom\_line() +}
\NormalTok{  ylab("Proportion of US Population Identifying As...") +}
\NormalTok{  xlab("") +}
\NormalTok{  labs(title = "Religious Affiliation in the United States",}
\NormalTok{       subtitle = "General Social Survey",}
\NormalTok{       color = "Religious Tradition",}
\NormalTok{      caption = "http://www.davideagle.org; Data: https://gss.norc.org") +}
\NormalTok{  theme\_minimal() + scFill +}
\NormalTok{  theme(axis.text.x = element\_text(angle = 70, hjust = 1)) +}
\NormalTok{  ggtitle("Religious Affiliation in the United States 1972{-}2018")}

\NormalTok{ggsave("content/post/reltrad{-}in{-}the{-}us/reltradusgss.jpg")}
\end{Highlighting}
\end{Shaded}

\{\{\textless{} figure src=``reltradusgss.jpg'' \textgreater\}\}

A few notable trends are evident in this plot. First of all, the
proportion of Roman Catholics in the United States has stayed relatively
steady over this period. While Latin American immigration has brought a
lot of new Catholics to the United States, this has been offset by
losses among the non-hispanic Catholic population.

Second, there has been a very steep rise in the number of people
claiming no religious affiliation. Non-affiliates were a pretty steady
5-8\% of the US population from the early 1970s to the early 1990s. The
proportion of nones has climbed steeply since then, reaching a high of
25\% of the US population, making them the second largest ``religious''
group in the United States. They could very well surpass Catholics by
the time the next wave of GSS data are released.

The higher proportion Black Protestant in 1972 is probably due to some
sort of artifact in the GSS data. There seems to be a lower proportion
of Conservative Protestants that year as well. This blip is also found
on other plots of GSS reltrad data,
\href{https://thesocietypages.org/ccf/2014/07/09/religious-change/fig-2-religious-change/}{here},
for example.

Third, Mainline Protestants are in free fall. They are now about 12\% of
the US population, down about 20 points from their peak in the early
1970s. Conservative Protestants made some gains in the mid-1980s to
mid-1990s (probably due to people leaving mainline denominations over
cultural issues), but are also declining. They are down about 10 points
from their high in the early 1990s.

Also important is that the ``Other'' religion category is steadily
climbing, and has reached a high of about 5\% of the US population. The
proportion of the population identifying as Black Protestants remained
fairly steady over this period, but recently has begun to decline. This
is likely due to more Black Protestants identifying with Conservative
Protestant denominations rather that historically Black denominations.

\end{document}
